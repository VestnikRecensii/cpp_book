\documentclass[book.tex]{subfiles}

\begin{document}
\input{fontsize}

\chaptertitle{Указатели. Массивы. Ссылки.}

\maketitle

\section*{Указатели}

\fordummies{

Машинная память состоит из последовательности пронумерованных однобайтных ячеек. Любая переменная имеет адрес --- номер ячейки, с которой в памяти начинаются данные, связанные с этой переменной.

}

Указатель --- тип данных, представляющий собой адрес ячейки памяти. Указатель может содержать адрес какой-либо переменной или функции, корректный адрес, не содержащий переменной, или специальные предопределенные значения \cppword{NULL} или \cppword{nullptr}\footnote{Появился в стандарте C++11}, означающие пустой или нулевой указатель. Указатели применяются:

\begin{itemize}

\item Для передачи данных в функцию. Передача указателя на данные быстрее, чем копирование этих данных. Также через указатели можно передавать в функцию другие функции.

\item Для возврата данных из функции. C++ позволяет вернуть только одно значение с помощью \cppword{return}. Если нужно вернуть несколько значений, можно использовать параметры-указатели.

\item Для работы с массивами и другими сложными данными - списками, строками, файлами.

\item Для работы с динамической памятью.

Пора рассмотреть эти возможности на примере:

\showfile{src/pointers.cpp}{Указатели}

\end{itemize}

\section*{Массивы}

\section*{Ссылки}

\end{document}
