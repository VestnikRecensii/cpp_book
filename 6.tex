\documentclass[book.tex]{subfiles}

\begin{document}
\fontsize{14pt}{16pt}\selectfont  %set font size

\chaptertitle{Указатели. Массивы. Ссылки.}

\maketitle

\section*{Указатели}

\fordummies{

Машинная память состоит из последовательности пронумерованных однобайтных ячеек. Любая переменная имеет адрес --- номер ячейки, с которой в памяти начинаются данные, связанные с этой переменной.

}

Указатель --- тип данных, представляющий собой адрес ячейки памяти. Указатель может содержать адрес какой-либо переменной, корректный адрес, не содержащий переменной, или специальные предопределенные значения \cppword{NULL} и \cppword{nullptr} (последний появился в стандарте C++11), означающие пустой или нулевой указатель.

\section*{Массивы}

\section*{Ссылки}

\end{document}
