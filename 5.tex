\documentclass[book.tex]{subfiles}

%\usepackage [utf8] {inputenc}
\usepackage [T2A] {fontenc}
\usepackage[english, russian]{babel}
\usepackage[left=2cm,right=2cm, top=2cm,bottom=2cm,bindingoffset=0cm]{geometry} % page settings
\usepackage[usenames]{color}
\usepackage{colortbl}
\usepackage{xcolor}
\usepackage{framed}
\usepackage{caption,subcaption}
\usepackage{listingsutf8}
\usepackage{titling}
\usepackage{soulpos}
\usepackage{soul}

\definecolor{lightcyan}{rgb}{0.88, 1.0, 1.0}

\lstset{
	extendedchars=true,
	keepspaces=true,
	literate={а}{{\selectfont\char224}}1
           {б}{{\selectfont\char225}}1
           {в}{{\selectfont\char226}}1
           {г}{{\selectfont\char227}}1
           {д}{{\selectfont\char228}}1
           {е}{{\selectfont\char229}}1
           {ё}{{\"e}}1
           {ж}{{\selectfont\char230}}1
           {з}{{\selectfont\char231}}1
           {и}{{\selectfont\char232}}1
           {й}{{\selectfont\char233}}1
           {к}{{\selectfont\char234}}1
           {л}{{\selectfont\char235}}1
           {м}{{\selectfont\char236}}1
           {н}{{\selectfont\char237}}1
           {о}{{\selectfont\char238}}1
           {п}{{\selectfont\char239}}1
           {р}{{\selectfont\char240}}1
           {с}{{\selectfont\char241}}1
           {т}{{\selectfont\char242}}1
           {у}{{\selectfont\char243}}1
           {ф}{{\selectfont\char244}}1
           {х}{{\selectfont\char245}}1
           {ц}{{\selectfont\char246}}1
           {ч}{{\selectfont\char247}}1
           {ш}{{\selectfont\char248}}1
           {щ}{{\selectfont\char249}}1
           {ъ}{{\selectfont\char250}}1
           {ы}{{\selectfont\char251}}1
           {ь}{{\selectfont\char252}}1
           {э}{{\selectfont\char253}}1
           {ю}{{\selectfont\char254}}1
           {я}{{\selectfont\char255}}1
           {А}{{\selectfont\char192}}1
           {Б}{{\selectfont\char193}}1
           {В}{{\selectfont\char194}}1
           {Г}{{\selectfont\char195}}1
           {Д}{{\selectfont\char196}}1
           {Е}{{\selectfont\char197}}1
           {Ё}{{\"E}}1
           {Ж}{{\selectfont\char198}}1
           {З}{{\selectfont\char199}}1
           {И}{{\selectfont\char200}}1
           {Й}{{\selectfont\char201}}1
           {К}{{\selectfont\char202}}1
           {Л}{{\selectfont\char203}}1
           {М}{{\selectfont\char204}}1
           {Н}{{\selectfont\char205}}1
           {О}{{\selectfont\char206}}1
           {П}{{\selectfont\char207}}1
           {Р}{{\selectfont\char208}}1
           {С}{{\selectfont\char209}}1
           {Т}{{\selectfont\char210}}1
           {У}{{\selectfont\char211}}1
           {Ф}{{\selectfont\char212}}1
           {Х}{{\selectfont\char213}}1
           {Ц}{{\selectfont\char214}}1
           {Ч}{{\selectfont\char215}}1
           {Ш}{{\selectfont\char216}}1
           {Щ}{{\selectfont\char217}}1
           {Ъ}{{\selectfont\char218}}1
           {Ы}{{\selectfont\char219}}1
           {Ь}{{\selectfont\char220}}1
           {Э}{{\selectfont\char221}}1
           {Ю}{{\selectfont\char222}}1
           {Я}{{\selectfont\char223}}1
}

\lstdefinestyle{shared}
{
    numbers=left,
    numbersep=1em,
    frame=single,
    framesep=\fboxsep,
    framerule=\fboxrule,
    rulecolor=\color{red},
    xleftmargin=\dimexpr\fboxsep+\fboxrule\relax,
    xrightmargin=\dimexpr\fboxsep+\fboxrule\relax,
    breaklines=true,
    tabsize=2,
    columns=flexible,
}

\lstdefinestyle{cpp}
{
    style=shared,
    language={C++},
    %alsolanguage={[Sharp]C},
    basicstyle=\small\tt,
    keywordstyle=\color{blue},
    commentstyle=\color[rgb]{0.13,0.54,0.13},
    backgroundcolor=\color{lightcyan},
    stringstyle=\color{orange},
    showstringspaces=false,
    basicstyle=\fontsize{12}{14}\selectfont\ttfamily
}

\lstnewenvironment{cpp}
{\lstset{style=cpp}}
{}

\definecolor{shadecolor}{RGB}{225,241,250}
\title{Введение в C++}
\author{Ярослав Зотов}
\date{Февраль 2018}

\newcommand{\fordummies}[1]{
    \begin{snugshade*}
        #1
    \end{snugshade*}
} %Text format for dumiies

\newcommand{\cppword}[1]{\sethlcolor{lightcyan}\hl{#1}}

\newcommand{\chaptertitle}[1]{
	\setlength{\voffset}{-7ex}
	\title{\textbf{\Huge #1}}
	\author{\vspace{-1ex}}
	\date{\vspace{-7ex}}
}

\newcommand{\showfile}[2] {
	\lstset{
		language=C++,
		caption={#2}, 
		captionpos=b,
		style=cpp
	}
	\lstinputlisting{#1}
}

\begin{document}
\fontsize{14pt}{16pt}\selectfont  %set font size

\chaptertitle{Функции}

\maketitle

\section*{Что такое функции}

В математике функцией называется соответствие между элементами двух множеств, установленное по такому правилу, что каждому элементу одного множества ставится в соответствие некоторый элемент из другого множества. В программировании понятие функция имеет несколько иное значение --- это фрагмент программного кода, к которому можно обратиться из другого места программы. Функции также называют подпрограммами.

Применяются функции для того, чтобы уменьшить количество кода и повысить структурированность. Сравните следующие два фрагмента:

\showfile{src/func1.cpp}{Код без функций}

\showfile{src/func2.cpp}{Код с функциями}

Второй вариант намного нагляднее --- он явно более структурированный, его проще читать, а при написании точно не ошибешься при копировании одинаковых строк.

Правила хорошего тона при написании функций гласят следующее:

\begin{itemize}
\item Функция должна иметь емкое имя. \cppword{fact}, \cppword{factorial} --- хорошие имена для фунции вычисления факториала, а \cppword{f1}, \cppword{func}, \cppword{function} --- нет.
\item Функция должна помещаться на один экран. 20--50 строк --- отличное значение для длины функции. Больше --- хуже.
\item Функция должна выполнять только одно логическое действие. Рассмотрим такой пример. Приложение запрашивает у пользователя логин и пароль, связывается с сервером и, если ответ положительный, пропускает пользователя дальше. Написание функции по имени 

\cppword{askLoginSendRequestGetToken}, 

объединяющее все эти действия внутри себя --- это плохо. Лучше сделать функцию \cppword{authorization}, которая внутри вызывает другие функции, например, \cppword{askLogin} и \cppword{getToken}, каждая из которых в свою очередь делает что-то более мелкое --- рисует окно с просьбой ввести логин и пароль, отправляет их на сервер и получает от него ответ.
\end{itemize}

\section*{Параметры и возвращаемые значения}

Функции может сообщить вызвавшему ее коду результат своей работы. Этот результат называется возвращаемым значением. Тип этого значения указывается перед названием функции. например, функция для суммирования двух целых чисел вероятнее всего имеет целый возвращаемый тип:

\showfile{src/sum.cpp}{Сумма двух чисел}

Данные, передаваемые в функцию, называются ее параметрами или аргументами. Различают формальные параметры --- те, которые указываются в описании функции, и фактические --- аргумент, передаваемый в функцию при вызове. В предыдущем примере \cppword{a} и \cppword{b} --- это формальные параметры, а \cppword{value1} и \cppword{10} --- фактические.

Конечно, функция может использовать не только свои параметры, но и глобальные переменные. Однако, в таком случае функция будет недетерминированной --- она может выдавать разный результат при одинаковых входных данных. А если функция к тому же изменяет глобальную переменную, осуществляет операции ввода-вывода, вызывает обработчики исключений или как-то иначе изменяет среду выполнения, то говорят, что функция имеет побочный эффект.  По возможности избегайте подобного:

\showfile{src/bad_func.cpp}{Недетерминированные функции и побочный эффект}

\section*{Передача параметров по ссылке и значению}

\section*{Объявление и определение}

\section*{Рекурсия}

\section*{Перегрузка функций}

\section*{Некоторые стандартные функции}

\end{document}
