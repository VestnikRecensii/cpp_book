\documentclass[book.tex]{subfiles}

%\usepackage [utf8] {inputenc}
\usepackage [T2A] {fontenc}
\usepackage[english, russian]{babel}
\usepackage[left=2cm,right=2cm, top=2cm,bottom=2cm,bindingoffset=0cm]{geometry} % page settings
\usepackage[usenames]{color}
\usepackage{colortbl}
\usepackage{xcolor}
\usepackage{framed}
\usepackage{caption,subcaption}
\usepackage{listingsutf8}
\usepackage{titling}
\usepackage{soulpos}
\usepackage{soul}

\definecolor{lightcyan}{rgb}{0.88, 1.0, 1.0}

\lstset{
	extendedchars=true,
	keepspaces=true,
	literate={а}{{\selectfont\char224}}1
           {б}{{\selectfont\char225}}1
           {в}{{\selectfont\char226}}1
           {г}{{\selectfont\char227}}1
           {д}{{\selectfont\char228}}1
           {е}{{\selectfont\char229}}1
           {ё}{{\"e}}1
           {ж}{{\selectfont\char230}}1
           {з}{{\selectfont\char231}}1
           {и}{{\selectfont\char232}}1
           {й}{{\selectfont\char233}}1
           {к}{{\selectfont\char234}}1
           {л}{{\selectfont\char235}}1
           {м}{{\selectfont\char236}}1
           {н}{{\selectfont\char237}}1
           {о}{{\selectfont\char238}}1
           {п}{{\selectfont\char239}}1
           {р}{{\selectfont\char240}}1
           {с}{{\selectfont\char241}}1
           {т}{{\selectfont\char242}}1
           {у}{{\selectfont\char243}}1
           {ф}{{\selectfont\char244}}1
           {х}{{\selectfont\char245}}1
           {ц}{{\selectfont\char246}}1
           {ч}{{\selectfont\char247}}1
           {ш}{{\selectfont\char248}}1
           {щ}{{\selectfont\char249}}1
           {ъ}{{\selectfont\char250}}1
           {ы}{{\selectfont\char251}}1
           {ь}{{\selectfont\char252}}1
           {э}{{\selectfont\char253}}1
           {ю}{{\selectfont\char254}}1
           {я}{{\selectfont\char255}}1
           {А}{{\selectfont\char192}}1
           {Б}{{\selectfont\char193}}1
           {В}{{\selectfont\char194}}1
           {Г}{{\selectfont\char195}}1
           {Д}{{\selectfont\char196}}1
           {Е}{{\selectfont\char197}}1
           {Ё}{{\"E}}1
           {Ж}{{\selectfont\char198}}1
           {З}{{\selectfont\char199}}1
           {И}{{\selectfont\char200}}1
           {Й}{{\selectfont\char201}}1
           {К}{{\selectfont\char202}}1
           {Л}{{\selectfont\char203}}1
           {М}{{\selectfont\char204}}1
           {Н}{{\selectfont\char205}}1
           {О}{{\selectfont\char206}}1
           {П}{{\selectfont\char207}}1
           {Р}{{\selectfont\char208}}1
           {С}{{\selectfont\char209}}1
           {Т}{{\selectfont\char210}}1
           {У}{{\selectfont\char211}}1
           {Ф}{{\selectfont\char212}}1
           {Х}{{\selectfont\char213}}1
           {Ц}{{\selectfont\char214}}1
           {Ч}{{\selectfont\char215}}1
           {Ш}{{\selectfont\char216}}1
           {Щ}{{\selectfont\char217}}1
           {Ъ}{{\selectfont\char218}}1
           {Ы}{{\selectfont\char219}}1
           {Ь}{{\selectfont\char220}}1
           {Э}{{\selectfont\char221}}1
           {Ю}{{\selectfont\char222}}1
           {Я}{{\selectfont\char223}}1
}

\lstdefinestyle{shared}
{
    numbers=left,
    numbersep=1em,
    frame=single,
    framesep=\fboxsep,
    framerule=\fboxrule,
    rulecolor=\color{red},
    xleftmargin=\dimexpr\fboxsep+\fboxrule\relax,
    xrightmargin=\dimexpr\fboxsep+\fboxrule\relax,
    breaklines=true,
    tabsize=2,
    columns=flexible,
}

\lstdefinestyle{cpp}
{
    style=shared,
    language={C++},
    %alsolanguage={[Sharp]C},
    basicstyle=\small\tt,
    keywordstyle=\color{blue},
    commentstyle=\color[rgb]{0.13,0.54,0.13},
    backgroundcolor=\color{lightcyan},
    stringstyle=\color{orange},
    showstringspaces=false,
    basicstyle=\fontsize{12}{14}\selectfont\ttfamily
}

\lstnewenvironment{cpp}
{\lstset{style=cpp}}
{}

\definecolor{shadecolor}{RGB}{225,241,250}
\title{Введение в C++}
\author{Ярослав Зотов}
\date{Февраль 2018}

\newcommand{\fordummies}[1]{
    \begin{snugshade*}
        #1
    \end{snugshade*}
} %Text format for dumiies

\newcommand{\cppword}[1]{\sethlcolor{lightcyan}\hl{#1}}

\newcommand{\chaptertitle}[1]{
	\setlength{\voffset}{-7ex}
	\title{\textbf{\Huge #1}}
	\author{\vspace{-1ex}}
	\date{\vspace{-7ex}}
}

\newcommand{\showfile}[2] {
	\lstset{
		language=C++,
		caption={#2}, 
		captionpos=b,
		style=cpp
	}
	\lstinputlisting{#1}
}

\begin{document}
\fontsize{14pt}{16pt}\selectfont  %set font size

\chaptertitle{Скалярные типы}

\maketitle 

\section*{Целые и вещественные типы}

С математической точки зрения целые и вещественные числа принадлежат к бесконечным множествам. У целых нет минимального и максимального значения, а вещественные и того больше --- они могут содержать бесконечное количество значений даже на отрезке от 0 до 1. Для подробной теоретической информации лучше обратиться к теории множеств, мы же лучше рассмотрим ситуацию с практической точки зрения.

\fordummies{

Очевидно, что на реальном устройстве бесконечность реализовать проблематично, а работать с числами надо. Проблема была решена просто --- целые числа ограничили по размеру, а вещественные еще и по точности. Поэтому, если вы услышите упоминания короткого целого или вещестенного числа двойной точности, не пугайтесь. С этими ограничениями связаны пара неочевидных моментов:

\begin{enumerate}
\itemПри арифметических операциях с целыми числами может произойти переполнение --- результат может оказаться больше, чем максимальный размер данного типа.
\itemАрифметические операции с вещественными числами выполняются с определенной точностью и могут содержать погрешность. Поэтому сравнивать вещественные числа на строгое равенство не рекомендуется. Подробно можно посмотреть здесь --- \href{https://0.30000000000000004.com}{https://0.30000000000000004.com}
\end{enumerate}

}

\section*{Целые типы}

В C++ существует несколько стандартных целочисленных типов --- \cppword{bool}, \cppword{char}, \cppword{short}, \cppword{int}, \cppword{long}, \cppword{wchar\_t}, \cppword{char16\_t}\footnote[1]{Появились в стандарте C++11}, \cppword{char32\_t}\footnotemark[1]. Стандарт гарантирует следующее соотношение размеров:

\showfile{src/parts/int_size.cpp}{Размеры целых типов}

\cppword{short}, \cppword{long} и \cppword{long long} --- это сокращения для \cppword{short int}, \cppword{long int} и \cppword{long long int} соответственно. Можно писать и так, и так, смысл от этого не меняется. Также у целого типа есть два модификатора знаковости: знаковый \cppword{signed} (позволяет оперировать числами со знаком) и беззнаковый \cppword{unsigned}, предназначенный для работы с неотрицательными числами. Стандарт гарантирует, что:

\begin{itemize}
\item \cppword{int} имеет размер не меньше 16 бит. Обычно он равен 4 байтам и может содержать значения от -2147483648 до 2147483647 для знакового или от 0 до 4294967295 для беззнакового типа.
\item \cppword{short} имеет размер не меньше 16 бит. Обычно он равен 2 байтам и может содержать значения от -32768 до 32767 для знакового или от до 65535 для беззнакового типа.
\item \cppword{long} имеет размер не меньше 32 бит. Он равен либо 4 байтам как \cppword{int}, либо 8 байтам, и тогда его пределы от $-2^{63}$ до $2^{63} - 1$ или от 0 до $2^{64} - 1$ для беззнакового типа.
\item \cppword{long long} имеет размер не меньше 64 бит. Обычно это 8 байт и тогда его пределы аналогичны типу \cppword{long}.
\end{itemize}

\fordummies{
Чаще всего пользуются знаковыми типами, но иногда применять беззнаковые логичнее. Например, возраст, длина и время как правило больше нуля, поэтому для них разумно использовать \cppword{unsigned}.
}

Вот что можно делать с целыми переменными:

\showfile{src/int_operations.cpp}{Операции с целыми типами}

\section*{Вещественные типы}

Cуществует 3 станадртных вещественных типа --- \cppword{float}, \cppword{double}, \cppword{long double}. По аналогии с целыми они отличаются размерами. Верно, что \cppword{float} $\leq$ \cppword{double} $\leq$ \cppword{long double}. Обычно они имеют следующие ограничения:

\begin{itemize}
\item \cppword{float} имеет размер 32 бит и точность порядка 7 знаков после запятой.
\item \cppword{double} имеет размер 64 бит и точность порядка 15 знаков после запятой.
\item \cppword{long double} может иметь размер 64, 80 или 128 бит и точность выше, чем у \cppword{double}.
\end{itemize}

К вещественным типам применимы те же операции, что и к цеылмм кроме поразрядных, нахождения остатка \cppword{\%}, \cppword{\%=}, инкремента \cppword{++} и декремента \cppword{--}. Операции деления \cppword{/} и \cppword{/=} делят без отбрасывания остатка.

Литералы вещестенного типа задаются 2 способами:

\begin{itemize}
\item Обыкновенным --- \cppword{0.1}, \cppword{1.5}. Если целая или дробнаячасть равна нулю, ее можно не указывать --- \cppword{.3} (0.3), \cppword{1.} (1.0).
\item С указанием порядка --- \cppword{5e-1} ($5*10^{-1}$), \cppword{5E-1} ($5*10^{-1}$), \cppword{50e-2} ($50*10^{-2}$), \cppword{0.5e0} ($0.5*10^{0}$) и \cppword{.5e0} ($0.5*10^{0}$). Порядок указывается как маленькой \cppword{e}, так и большой \cppword{E}. Ноль так же можно опускать.
\end{itemize}

Для точного указания принадлежности литерала к типу \cppword{long double} используют суффиксы \cppword{l} и \cppword{L}, а для \cppword{float} --- \cppword{f} и \cppword{F}:

\showfile{src/parts/float_literals.cpp}{Вещественные литералы}

\section*{Перечисления}

Часто приходится работать с ограниченным набором значений. Например, дней недели всего 7, месяцев 12, а основных цвета, используемых в отрисовке, и вовсе три (RGB --- красный, зеленый и синий). Конечно, можно в таком случае взять обычные целые числа и принять, что 0 --- это понедельник, 1 --- вторник и т.д., но это не очень наглядно. В таком случае удобно использовать перечисления.

Перечисления похожи на целый тип и представляют собой множество констант типа \cppword{int}. Значение каждой следующей константы равно значению предыдущей плюс 1. Нумерация начинается с нуля. Кроме того, константам можно явно присвоить значение.

\showfile{src/enum.cpp}{Использование перечислений}

На этом знакомство со скалярными типами заканчивается. Пришло время перейти к вещам, привычным для программистов, работающих с любыми языками программирования, --- блокам, условиям и ветвлениям.

\end{document}
